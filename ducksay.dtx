% \iffalse meta-comment
%
% File: ducksay.dtx Copyright (C) 2018 Jonathan P. Spratte
%
% This work  may be  distributed and/or  modified under  the conditions  of the
% LaTeX Project Public License (LPPL),  either version 1.3c  of this license or
% (at your option) any later version.  The latest version of this license is in
% the file:
%
%   http://www.latex-project.org/lppl.txt
%
% ------------------------------------------------------------------------------
%
%<*driver>
\def\nameofplainTeX{plain}
\ifx\fmtname\nameofplainTeX\else
  \expandafter\begingroup
\fi
\input l3docstrip.tex
\askforoverwritefalse
\preamble

--------------------------------------------------------------
ducksay -- cowsay for LaTeX
E-mail: jspratte@yahoo.de
Released under the LaTeX Project Public License v1.3c or later
See http://www.latex-project.org/lppl.txt
--------------------------------------------------------------

Copyright (C) 2017-2018 Jonathan P. Spratte

This  work may be  distributed and/or  modified under  the conditions  of the
LaTeX Project Public License (LPPL),  either version 1.3c  of this license or
(at your option) any later version.  The latest version of this license is in
the file:

  http://www.latex-project.org/lppl.txt

This work is "maintained" (as per LPPL maintenance status) by
  Jonathan P. Spratte.

This work consists of the file  ducksay.dtx
and the derived files           ducksay.pdf
                                ducksay.sty and
                                ducksay.animals.tex.

\endpreamble
% stop docstrip adding \endinput
\postamble
\endpostamble
\generate{\file{ducksay.sty}{\from{ducksay.dtx}{pkg}}}
\generate{\file{ducksay.animals.tex}{\from{ducksay.dtx}{animals}}}
\ifx\fmtname\nameofplainTeX
  \expandafter\endbatchfile
\else
  \expandafter\endgroup
\fi
%</driver>
%
%<*driver>
\ProvidesFile{ducksay.dtx}
  [2018/08/25 cowsay for LaTeX]
\documentclass{l3doc}
\usepackage{ducksay}
\renewcommand*{\thefootnote}{\fnsymbol{footnote}}
\newcommand*{\anml}{\meta{animal}}
\newcommand*{\msg}{\meta{message}}
\usepackage{enumitem}
\newenvironment{options}
  {\begin{description}[style=nextline,font=\normalfont\ttfamily]}
  {\end{description}}
\makeatletter
\newcommand*{\availableAnimal}[1]{\@for\cs:=#1\do{%
  \ifx\cs\@empty\else%
    \rlap{\expandafter\ducksay\expandafter[\cs]{\cs}}\hfill\mbox{}\\[1ex]%
  \fi%
}}
\makeatother
\begin{document}
  \DocInput{ducksay.dtx}
\end{document}
%</driver>
%<*pkg>
\NeedsTeXFormat{LaTeX2e}
\RequirePackage{xparse,l3keys2e}

\def\ducksay@version{v1.2}
\def\ducksay@date{2017/10/30}

\ProvidesExplPackage
  {ducksay}           {\ducksay@date}
  {\ducksay@version}  {cowsay for LaTeX}
%</pkg>
%
% \begin{titlepage}^^A>>>
%   \makeatletter
%   \centering
%   %\mbox{}\vfill
%   \Large
%     \ducksay[duck,bubble=\huge,msg-align=c,wd=8]{This is\\ducksay!}\\
%   \vfill
%   \normalsize
%   \hspace*{-2cm}
%     \ducksay[cow,bubble=\large]{\ducksay@version}\\
%   \small
%   \vspace*{-5cm}\hspace*{5cm}
%     \ducksay[small-duck,bubble=\normalsize]{But which Version?}
%   \mbox{}\hfil
%   \vspace{2cm}
%   \vfill
%   \vfill
%   \hspace*{-0cm}
%   \large
%   \smash{%
%     \ducksay[r2d2,bubble=\large]{by Jonathan P. Spratte}}
%   \small
%     \ducksay[hedgehog,bubble=\normalsize]{Today is \ducksay@date}
%   \makeatother
% \end{titlepage}^^A<<<
% \tableofcontents
%
% \begin{documentation}
%
% \section{Macros}^^A>>>
% \marginpar{%
%   \rlap{%
%     \tiny\ducksay[yoda,bubble=\footnotesize,align=t]{Use those, you might}}}
% The following macros are available:
%
% \begin{function}{\ducksay}^^A>>>
%   \begin{syntax}
%     \cs{ducksay}\oarg{options}\marg{message}
%   \end{syntax}
%   options might include any of the options described in
%   \autoref{sec:options}. Prints an \anml\ saying \msg. \msg\ is not read in
%   verbatim. Multi-line \msg s are possible using |\\|. |\\| should not be
%   inside a macro but at toplevel. Else use the option |ht|.
% \end{function}^^A<<<
%
% \begin{function}{\duckthink}^^A>>>
%   \begin{syntax}
%     \cs{duckthink}\oarg{options}\marg{message}
%   \end{syntax}
%   options might include any of the options described in
%   \autoref{sec:options}. Prints an \anml\ thinking \msg. \msg\ is not read in
%   verbatim. It is implemented using regular expressions replacing a |\|
%   which is only preceded by |\s*| in the first three lines with |O|
%   and |o|. It is therefore slower than \cs{ducksay}. Multi-line \msg s
%   are possible using |\\|. |\\| should not be inside a macro but at
%   toplevel. Else use the option |ht|.
% \end{function}^^A<<<
%
% \begin{function}{\DefaultAnimal}^^A>>>
%   \begin{syntax}
%     \cs{DefaultAnimal}\marg{animal}
%   \end{syntax}
%   use the \anml\ if none is given in the optional argument to \cs{ducksay}
%   or \cs{duckthink}. Package default is |duck|.
% \end{function}^^A<<<
%
% \begin{function}{\DucksayOptions}^^A>>>
%   \begin{syntax}
%     \cs{DucksayOptions}\marg{options}
%   \end{syntax}
%   set the defaults to the keys described in \autoref{sec:options}. Don't use
%   an \anml\ here, it has no effect.
% \end{function}^^A<<<
%
% \begin{function}{\AddAnimal}^^A>>>
%   \begin{syntax}
%     \cs{DefaultAnimal}\meta{*}\marg{animal}\meta{ascii-art}
%   \end{syntax}
%   adds \anml\ to the known animals. \meta{ascii-art} is multi-line verbatim
%   and therefore should be delimited either by matching braces or by anything
%   that works for \cs{verb}. If the star is given \anml\ is the new default.
%   One space is added to the begin of \anml\ (compensating the opening symbol).
%   For example, snowman is added with:\\[1ex]
%   \begin{minipage}{\linewidth}
% \begin{verbatim}
% \AddAnimal{snowman}
% {  \
%     \_[_]_
%       (")
%    >-( : )-<
%     (__:__)}
% \end{verbatim}
%   \end{minipage}
% \end{function}^^A<<<
%^^A<<<
%
% \section{Options}\label{sec:options}^^A>>>
% \begingroup
%   \reversemarginpar
%   \marginpar
%     {%
%       \vspace*{-2em}\hspace*{-4em}\tiny
%       \ducksay[hedgehog,bubble=\footnotesize,align=t]
%         {Everyone likes\\options}%
%     }
% \endgroup
% The following options are available to \cs{ducksay}, \cs{duckthink}, and
% \cs{DucksayOptions} and if not otherwise specified also as package options:
% \begin{options}
%   \item[\anml] 
%     One of the animals listed in \autoref{sec:animals} or any of the ones
%     added with \cs{AddAnimal}. Not useable as package option.
%   \item[animal=\anml]
%     a longer alternative to the use of \anml\ if used in \cs{ducksay} or
%     \cs{duckthink}. If it is used as a package option or in
%     \cs{DucksayOptions} it changes the default animal to \anml.
%   \item[bubble=\meta{code}]
%     use \meta{code} in a group right before the bubble (for font switches).
%     Might be used as a package option but not all control sequences work out
%     of the box there.
%   \item[body=\meta{code}]
%     use \meta{code} in a group right before the body (meaning the \anml).
%     Might be used as a package option but not all control sequences work out
%     of the box there. E.g., to right-align the \anml\ to the bubble, use
%     \verb|body=\hfill|.
%   \item[align=\meta{valign}]
%     use \meta{valign} as the vertical alignment specifier given to the
%     \env{tabular} which is around the contents of \cs{ducksay} and
%     \cs{duckthink}.
%   \item[msg-align=\meta{halign}]
%     use \meta{halign} for alignment of the rows of multi-line \msg s. It
%     should match a \texttt{tabular} column specifier. Default is |l|. It only
%     affects the contents of the speech bubble not the bubble.
%   \item[rel-align=\meta{column}]
%     use \meta{column} for alignment of the bubble and the body. It should
%     match a \env{tabular} column specifier. Default is |l|.
%   \item[wd=\meta{count}]
%     in order to detect the width the \msg\ is expanded. This might not work
%     out for some commands (e.g. \cs{url} from \pkg{hyperref}). If you
%     specify the width using |wd| the \msg\ is not expanded and
%     therefore the command \emph{might} work out. \meta{count} should be the
%     character count.
%   \item[ht=\meta{count}]
%     you might explicitly set the height (the row count) of the \msg. This only
%     has an effect if you also specify |wd|.
%   \item[ligatures=\meta{regex}]
%     this is a \LaTeX3 regular expression which should match every character
%     you don't want to form ligatures during \cs{AddAnimal}. The default
%     expression is |[`<>,'\-]|. Giving no argument (or an empty one) disables
%     the replacement, which enhances compilation speed. The formation of
%     ligatures was only observed in combination with
%     \verb|\usepackage[T1]{fontenc}| by the author of this package. Therefore
%     giving the option |ligatures| without an argument might enhance the
%     compilation speed for you without any drawbacks.
% \end{options}
%^^A<<<
%
% \section{Defects}^^A>>>
% \begingroup
%   \reversemarginpar
%   \marginpar
%     {%
%       \tiny\rlap{\ducksay[frog,bubble=\footnotesize,align=t]{Ohh, no!}}%
%     }
% \endgroup
% \begin{itemize}
%   \item no automatic line wrapping
% \end{itemize}^^A<<<
%
% \section{Dependencies}^^A>>>
% \marginpar
%   {%
%     \tiny
%     \rlap{\ducksay[kangaroo,bubble=\footnotesize,align=t]{We rely on you}}%
%   }
% The package depends on the two packages \pkg{xparse} and \pkg{l3keys2e}
% and all of their dependencies.
%^^A<<<
%
% \section{Available Animals}\label{sec:animals}^^A>>>
% The following animals are provided by this package. I did not create them (but
% altered some), they belong to their original creators.
% \bgroup
% \footnotesize
% \begin{multicols}{2}
% \availableAnimal{%>>>
%   ,duck%
%   ,small-duck%
%   ,duck-family%
%   ,small-rabbit%
%   ,squirrel%
%   ,cow%
%   ,tux%
%   ,head-in%
%   ,pig%
%   ,frog%
%   ,snowman%
%   ,bunny%
%   ,dragon%
%   ,sodomized%
%   ,hedgehog%
%   ,kangaroo%
%   ,dog%
%   ,rabbit%
%   ,unicorn%
%   ,snail%
%   ,whale%
% }\end{multicols}\begin{multicols}{2}
% \availableAnimal{%
%   ,r2d2%
%   ,vader%
%   ,yoda-head%
%   ,small-yoda%
%   ,yoda%
% }%<<<
% \end{multicols}
% \egroup
%^^A<<<
%
% \section{Miscellaneous}^^A>>>
% \marginpar
%   {%
%     \rlap
%       {%
%         \tiny
%         \ducksay[squirrel,bubble=\footnotesize,align=t]{I'd choose WTFPL}%
%       }%
%   }
% This package is distributed under the terms of the GPLv3 or later, or the LPPL
% 1.3c or later, choose which ever license fits your needs the best.
%
% The package is hosted on \url{https://github.com/Skillmon/ltx_ducksay}, you
% might report bugs there.
%^^A<<<
%
% \clearpage
%^^A closing page>>>
% \thispagestyle{empty}
% \bgroup
% \Huge
% \mbox{}\vfill
% \centering
% \makebox[0pt]{\duckthink{Who's gonna use it anyway?}}
% \vfill
% \hfill\smash{%
%   \footnotesize\ducksay[small-yoda,wd=39,ht=3,msg-align=c,rel-align=r]
%     {Hosted at\\\url{https://github.com/Skillmon/ltx_ducksay}\\it is.}}
% \egroup
%^^A<<<
%
% \end{documentation}
%
% \begin{implementation}
%
%    \begin{macrocode}
%<*pkg>
%    \end{macrocode}
% 
%    \begin{macrocode}
%<@@=ducksay>
%    \end{macrocode}
%
% \section{Variables}^^A>>>
% \begin{variable}{\l_ducksay_strlen_int}^^A>>>
%    \begin{macrocode}
\int_new:N \l_ducksay_strlen_int
%    \end{macrocode}
% \end{variable}^^A<<<
% \begin{variable}{\l_ducksay_lines_int}^^A>>>
%    \begin{macrocode}
\int_new:N \l_ducksay_lines_int
%    \end{macrocode}
% \end{variable}^^A<<<
% \begin{variable}{\l_ducksay_msg_lines_seq}^^A>>>
%    \begin{macrocode}
\seq_new:N \l_ducksay_msg_lines_seq
%    \end{macrocode}
% \end{variable}^^A<<<
% \begin{variable}{\l_ducksay_align_tl}^^A>>>
%    \begin{macrocode}
\tl_new:N \l_ducksay_align_tl
%    \end{macrocode}
% \end{variable}^^A<<<
% \begin{variable}{\l_ducksay_msg_align_tl}^^A>>>
%    \begin{macrocode}
\tl_new:N \l_ducksay_msg_align_tl
%    \end{macrocode}
% \end{variable}^^A<<<
% \begin{variable}{\l_ducksay_animal_tl}^^A>>>
%    \begin{macrocode}
\tl_new:N \l_ducksay_animal_tl
%    \end{macrocode}
% \end{variable}^^A<<<
% \begin{variable}{\g__ducksay_all_animals_clist}^^A>>>
%    \begin{macrocode}
\clist_new:N \g__ducksay_all_animals_clist
%    \end{macrocode}
% \end{variable}^^A<<<
% \begin{variable}{\l_ducksay_empty_ligatures_bool}^^A>>>
%    \begin{macrocode}
\bool_new:N \l_ducksay_empty_ligatures_bool
%    \end{macrocode}
% \end{variable}^^A<<<
% \begin{variable}{\l_ducksay_auto_bool}^^A>>>
%    \begin{macrocode}
\bool_new:N \l_ducksay_auto_bool
%    \end{macrocode}
% \end{variable}^^A<<<
% \begin{variable}{\ducksay_bubble:}^^A>>>
%    \begin{macrocode}
\cs_new:Nn \ducksay_bubble: {}
%    \end{macrocode}
% \end{variable}^^A<<<
% \begin{variable}{\ducksay_body:}^^A>>>
%    \begin{macrocode}
\cs_new:Nn \ducksay_body: {}
%    \end{macrocode}
% \end{variable}^^A<<<
% \begin{variable}{\l_ducksay_ligatures_regex}^^A>>>
%    \begin{macrocode}
\regex_new:N \l_ducksay_ligatures_regex
%    \end{macrocode}
% \end{variable}^^A<<<
% \begin{variable}{\l_ducksay_xwidth_dim}^^A>>>
%    \begin{macrocode}
\dim_new:N \l_ducksay_xwidth_dim
%    \end{macrocode}
% \end{variable}^^A<<<
% \subsection{For Temporary Usage}
% \begin{variable}{\l_ducksay_tmpa_box}^^A>>>
%    \begin{macrocode}
\box_new:N \l_ducksay_tmpa_box
%    \end{macrocode}
% \end{variable}^^A<<<
% \begin{variable}{\l_ducksay_tmpa_tl}^^A>>>
%    \begin{macrocode}
\tl_new:N  \l_ducksay_tmpa_tl
%    \end{macrocode}
% \end{variable}^^A<<<
^^A<<<
%
% \section{Regular Expressions}^^A>>>
% Regular expressions for \cs{duckthink}
%    \begin{macrocode}
\regex_const:Nn \c_ducksay_first_regex  { \A(.\s*)\\ }
\regex_const:Nn \c_ducksay_second_regex { \A(.[^\c{null}]*\c{null}\s*)\\ }
\regex_const:Nn \c_ducksay_third_regex  {
  \A(.[^\c{null}]*\c{null}[^\c{null}]*\c{null}\s*)\\ }
%    \end{macrocode}
% Regular expressions for \cs{AddAnimal}
%    \begin{macrocode}
\regex_const:Nn \c_ducksay_newline_regex { \r }
\regex_set:Nn \l_ducksay_ligatures_regex { [`<>,'\-] }
%    \end{macrocode}
^^A<<<
%
% \section{Messages}^^A>>>
%    \begin{macrocode}
\msg_new:nnnn { ducksay } { option-unknown }
  { Unknown\ option\ '#1'\ for\ package\ ducksay. }
  { If\ the\ option\ corresponds\ to\ one\ of\ the\ animals,\ make\ sure\ to\ 
    load\ that\ animal. }
%    \end{macrocode}^^A<<<
%
% \section{Key-value setup}^^A>>>
%    \begin{macrocode}
\keys_define:nn { ducksay }^^A>>>
  {
    ,bubble .code:n       = \cs_set:Nn \ducksay_bubble: {#1}
    ,body   .code:n       = \cs_set:Nn \ducksay_body: {#1}
    ,align  .tl_set:N     = \l_ducksay_align_tl
    ,align  .value_required:n = true
    ,wd     .int_set:N    = \l_ducksay_strlen_int
    ,wd     .value_required:n = true
    ,ht     .int_set:N    = \l_ducksay_lines_int
    ,ht     .value_required:n = true
    ,animal .code:n       = {
      \keys_define:nn { ducksay } { default_animal .meta:n = {#1} }}
    ,animal .initial:n    = duck
    ,msg-align .tl_set:N  = \l_ducksay_msg_align_tl
    ,msg-align .initial:n = l
    ,msg-align .value_required:n = true
    ,rel-align .tl_set:N  = \l_ducksay_rel_align_tl
    ,rel-align .initial:n = l
    ,rel-align .value_required:n = true
    ,auto   .bool_set:N   = \l_ducksay_auto_bool
    ,WD     .meta:n       = { auto = true , wd = #1 }
    ,ligatures .code:n    = {
      \tl_if_empty:nTF { #1 }
        { \bool_set_true:N \l_ducksay_empty_ligatures_bool }
        {
          \bool_set_false:N \l_ducksay_empty_ligatures_bool
          \regex_set:Nn \l_ducksay_ligatures_regex { #1 }
        }
    }
    ,unknown   .code:n    = {
      \msg_error:nnx { ducksay } { option-unknown } { \l_keys_key_tl }
    }
  }^^A<<<
%    \end{macrocode}
% Setting up a clist of all included animals^^A>>>
%    \begin{macrocode}
\clist_set:Nn \g__ducksay_all_animals_clist
  {
    duck
    ,small-duck
    ,duck-family
    ,squirrel
    ,cow
    ,head-in
    ,sodomized
    ,tux
    ,pig
    ,frog
    ,snowman
    ,bunny
    ,small-rabbit
    ,rabbit
    ,dragon
    ,hedgehog
    ,kangaroo
    ,dog
    ,snail
    ,unicorn
    ,whale
    ,r2d2
    ,vader
    ,yoda-head
    ,small-yoda
    ,yoda
  }
%    \end{macrocode}
% define load-switches for each of them
%    \begin{macrocode}
\clist_map_inline:Nn \g__ducksay_all_animals_clist
  {
    \bool_new:c { g__ducksay_load_#1_bool }
    \keys_define:nn { ducksay }
      { #1 .bool_gset:c = { g__ducksay_load_#1_bool } }
  }
%    \end{macrocode}
% define one key to call of them
%    \begin{macrocode}
\keys_define:nn { ducksay }
  {
    ,all .code:n =
      {
        \tl_if_empty:nTF { #1 }
          {
            \clist_map_inline:Nn \g__ducksay_all_animals_clist
              { \keys_set:nn { ducksay } { ##1 } }
          }
          {
            \clist_map_inline:Nn \g__ducksay_all_animals_clist
              { \keys_set:nn { ducksay } { ##1 = #1 } }
          }
      }
  }
%    \end{macrocode}^^A<<<
% by default load |duck|
%    \begin{macrocode}
\keys_set:nn { ducksay } { duck }
%    \end{macrocode}
%
%    \begin{macrocode}
\ProcessKeysOptions { ducksay }
%    \end{macrocode}
%
% Undefine the |all| key
%    \begin{macrocode}
\keys_define:nn { ducksay } { ,all .undefine: }
%    \end{macrocode}^^A<<<

\cs_new:Nn \ducksay_x_dims:%>>>
  {
    \hbox_set:Nn \l_ducksay_tmpa_box { x }
    \dim_set:Nn \l_ducksay_xwidth_dim { \box_wd:N \l_ducksay_tmpa_box }
  }
%<<<

% \section{Functions}
% \subsection{Internal}^^A>>>
%
% \begin{macro}{\ducksay_longest_line:n}^^A>>>
%   Calculate the length of the longest line
%    \begin{macrocode}
\cs_new:Nn \ducksay_longest_line:n
  {
    \int_incr:N \l_ducksay_lines_int
    \exp_args:NNx \tl_set:Nn \l_ducksay_tmpa_tl { #1 }
    \regex_replace_all:nnN { \s } { \c{space} } \l_ducksay_tmpa_tl
    \int_set:Nn \l_ducksay_strlen_int {
      \int_max:nn { \l_ducksay_strlen_int } { \tl_count:N \l_ducksay_tmpa_tl } }
  }
%    \end{macrocode}
% \end{macro}^^A<<<
%
% \begin{macro}{\ducksay_open_bubble:}^^A>>>
%   Draw the opening bracket of the bubble
%    \begin{macrocode}
\cs_new:Nn \ducksay_open_bubble:
  {
    \begin{tabular}{@{}l@{}}
      \mbox{}\\
      \int_compare:nNnTF {\l_ducksay_lines_int} = {1} { ( }
        {
          /
          \int_step_inline:nnn { 3 } { \l_ducksay_lines_int } { \\\kern-0.5ex| }
          \\\detokenize{\ }
        }
      \\[-1ex]\mbox{}
    \end{tabular}
    \begin{tabular}{@{}l@{}}
      _\\
      \int_step_inline:nnn { 2 } { \l_ducksay_lines_int } { \\ } \\[-1ex]
      \mbox{-}
    \end{tabular}
  }
%    \begin{macrocode}
% \end{macro}^^A<<<
%
% \begin{macro}{\ducksay_close_bubble:}^^A>>>
%   Draw the closing bracket of the bubble
%    \begin{macrocode}
\cs_new:Nn \ducksay_close_bubble:
  {
    \begin{tabular}{@{}l@{}}
      _\\
      \int_step_inline:nnn { 2 } { \l_ducksay_lines_int } { \\ } \\[-1ex]
      {-}
    \end{tabular}
    \begin{tabular}{@{}r@{}}
      \mbox{}\\
      \int_compare:nNnTF {\l_ducksay_lines_int} = {1} { ) }
        {
          \detokenize{\ }
          \int_step_inline:nnn { 3 } { \l_ducksay_lines_int } { \\|\kern-0.5ex }
          \\/
        }
      \\[-1ex]\mbox{}
    \end{tabular}
  }
%    \begin{macrocode}
% \end{macro}^^A<<<
%
% \begin{macro}{\ducksay_print_msg:nn}^^A>>>
%   Print out the message
%    \begin{macrocode}
\cs_new:Nn \ducksay_print_msg:nn
  {
    \begin{tabular}{@{} #2 @{}}
      \int_step_inline:nn { \l_ducksay_strlen_int } { _ } \\
      #1\\[-1ex]
      \int_step_inline:nn { \l_ducksay_strlen_int } { {-} }
    \end{tabular}
  }
\cs_generate_variant:Nn \ducksay_print_msg:nn { nV }
%    \begin{macrocode}
% \end{macro}^^A<<<
%
% \begin{macro}{\ducksay_print:nn}^^A>>>
%   Print out the whole thing
%    \begin{macrocode}
\cs_new:Nn \ducksay_print:nn
  {
    \int_compare:nNnTF { \l_ducksay_strlen_int } = { 0 }
      {
        \seq_set_split:Nnn \l_ducksay_msg_lines_seq {\\} { #1 }
        \seq_map_function:NN \l_ducksay_msg_lines_seq \ducksay_longest_line:n
      }
      {
        \int_compare:nNnT { \l_ducksay_lines_int } = { 0 }
          {
            \regex_count:nnN {\c{\\}} {#1} \l_ducksay_lines_int
            \int_incr:N \l_ducksay_lines_int
          }
      }
    \group_begin:
      \frenchspacing
      \verbatim@font
      \@noligs
      \begin{tabular}[\l_ducksay_align_tl]{@{}#2@{}}
        \ducksay_bubble:
        \begin{tabular}{@{}l@{}}
          \ducksay_open_bubble:
          \ducksay_print_msg:nV {#1} \l_ducksay_msg_align_tl
          \ducksay_close_bubble:
        \end{tabular}\\
        \ducksay_body:
        \begin{tabular}{@{}l@{}}
          \l_ducksay_animal_tl
        \end{tabular}
      \end{tabular}
    \group_end:
  }
\cs_generate_variant:Nn \ducksay_print:nn { nV }
%    \begin{macrocode}
% \end{macro}^^A<<<
%
% \begin{macro}{\ducksay_prepare_say_and_think:n}^^A>>>
%   Reset some variables
%    \begin{macrocode}
\cs_new:Nn \ducksay_prepare_say_and_think:n
  {
    \int_zero:N \l_ducksay_strlen_int
    \int_zero:N \l_ducksay_lines_int
    \keys_define:nn { ducksay } { animal .meta:n = { ##1 } }
    \keys_set:nn { ducksay } { default_animal , #1 }
  }
%    \begin{macrocode}
% \end{macro}^^A<<<
%
% \begin{macro}{\ducksay_to_duckthink:}^^A>>>
%   Convert ducksay to duckthink
%    \begin{macrocode}
\cs_new:Nn \ducksay_to_duckthink:
  {
    \regex_replace_once:NnN \c_ducksay_first_regex  { \1O } \l_ducksay_animal_tl
    \regex_replace_once:NnN \c_ducksay_second_regex { \1o } \l_ducksay_animal_tl
    \regex_replace_once:NnN \c_ducksay_third_regex  { \1o } \l_ducksay_animal_tl
  }
%    \begin{macrocode}
% \end{macro}^^A<<<
%^^A<<<
%
% \subsection{Document level}
%
% \begin{macro}{\ducksay}^^A>>>
%    \begin{macrocode}
\NewDocumentCommand \ducksay { O{} m }
  {
    \group_begin:
      \ducksay_prepare_say_and_think:n { #1 }
      \ducksay_print:nV { #2 } \l_ducksay_rel_align_tl
    \group_end:
  }
%    \begin{macrocode}
% \end{macro}^^A<<<
%
% \begin{macro}{\duckthink}^^A>>>
%    \begin{macrocode}
\NewDocumentCommand \duckthink { O{} m }
  {
    \group_begin:
      \ducksay_prepare_say_and_think:n { #1 }
      \ducksay_to_duckthink:
      \ducksay_print:nV { #2 } \l_ducksay_rel_align_tl
    \group_end:
  }
%    \begin{macrocode}
% \end{macro}^^A<<<
%
% \begin{macro}{\DefaultAnimal}^^A>>>
%    \begin{macrocode}
\NewDocumentCommand \DefaultAnimal { m }
  {
    \keys_define:nn { ducksay } { default_animal .meta:n = {#1} }
  }
%    \begin{macrocode}
% \end{macro}^^A<<<
%
% \begin{macro}{\DucksayOptions}^^A>>>
%    \begin{macrocode}
\NewDocumentCommand \DucksayOptions { m }
  {
    \keys_set:nn { ducksay } { #1 }
  }
%    \begin{macrocode}
% \end{macro}^^A<<<
%
% \begin{macro}{\AddAnimal}^^A>>>
%    \begin{macrocode}
\NewDocumentCommand \AddAnimal { s m +v }
  {
    \tl_set:Nn \l_ducksay_tmpa_tl { \ #3 }
    \bool_if:NF \l_ducksay_empty_ligatures_bool
      {
        \regex_replace_all:NnN \l_ducksay_ligatures_regex { \c{mbox}\0 }
          \l_ducksay_tmpa_tl 
      }
    \regex_replace_all:NnN
      \c_ducksay_newline_regex { \c{tabularnewline}\c{null} } \l_ducksay_tmpa_tl
    \tl_gset_eq:cN { g_ducksay_animal_#2_tl } \l_ducksay_tmpa_tl
    \keys_define:nn { ducksay }
      {
        #2 .code:n =
          \tl_set_eq:Nc \l_ducksay_animal_tl { g_ducksay_animal_#2_tl }
      }
    \IfBooleanT{#1}
      {
        \keys_define:nn { ducksay } { default_animal .meta:n = { #2 } }
      }
  }
%    \begin{macrocode}
% \end{macro}^^A<<<

\NewDocumentCommand{\ducksay@AddAnimal}{ s m +v }
  {
    \bool_if:cT { g__ducksay_load_#2_bool }
      {
        \tl_set:Nn \l_ducksay_tmpa_tl { \ #3 }
        \bool_if:NF \l_ducksay_empty_ligatures_bool
          {
            \regex_replace_all:NnN
              \l_ducksay_ligatures_regex { \c{mbox}\0 } \l_ducksay_tmpa_tl 
          }
        \regex_replace_all:NnN \c_ducksay_newline_regex
          { \c{tabularnewline}\c{null} } \l_ducksay_tmpa_tl
        \tl_gset_eq:cN { g_ducksay_animal_#2_tl } \l_ducksay_tmpa_tl
        \keys_define:nn { ducksay }
          {
            #2 .undefine:,
            #2 .code:n =
              \tl_set_eq:Nc \l_ducksay_animal_tl { g_ducksay_animal_#2_tl }
          }
        \IfBooleanT{#1}
          {
            \keys_define:nn { ducksay } { default_animal .meta:n = { #2 } }
          }
      }
  }
\ExplSyntaxOff
\input{ducksay.animals.tex}
%
%    \begin{macrocode}
\endinput
%    \end{macrocode}
%
% \end{implementation}
%
%    \begin{macrocode}
%</pkg>
%    \end{macrocode}
%
%<*animals>^^A>>>
^^A some of the below are from http://ascii.co.uk/art/kangaroo
\ducksay@AddAnimal{duck}^^A>>>
{  \
    \   __
      >(' )
        )/
       /(
      /  `----/
      \  ~=- /
    ~^~^~^~^~^~^~^}^^A<<<
\ducksay@AddAnimal{small-duck}^^A>>>
{  \
    \
      >()_
       (__)__ _}^^A<<<
\ducksay@AddAnimal{duck-family}^^A>>>
{  \
    \   __
      >(' )
        )/
       /(
      /  `----/  -()_  >()_
    __\__~=-_/__ _(__)__(__)__ _}^^A<<<
\ducksay@AddAnimal{cow}^^A>>>
{  \  ^__^
    \ (oo)\_______
      (__)\       )\/\
          ||----w |
          ||     ||}^^A<<<
\ducksay@AddAnimal{head-in}^^A>>>
{  \  
    \ ^__^         /
      (oo)\_______/  ________
      (__)\       )=(  ___|_ \____
          ||----w |  \ \    \____ |
          ||     ||   ||         ||}^^A<<<
\ducksay@AddAnimal{sodomized}^^A>>>
{  \             _
    \           (_)
      ^__^       / \
      (oo)\_____/_\ \
      (__)\       ) /
          ||----w ((
          ||     ||>>}^^A<<<
\ducksay@AddAnimal{tux}^^A>>>
{  \
    \  .--. 
      |o_o |
      |\_/ |
     //   \ \
    (|     | )
   /'\_   _/`\
   \___)=(___/}^^A<<<
\ducksay@AddAnimal{pig}^^A>>>
+  \     _//| .-~~~-.
    \ _/oo  }        }-@
     ('')_  }        |
      `--'| { }--{  }
           //_/  /_/+^^A<<<
\ducksay@AddAnimal{frog}^^A>>>
{   \
     \ (.)_(.)
    _ (   _   ) _
   / \/`-----'\/ \
 __\ ( (     ) ) /__
 )   /\ \._./ /\   (
  )_/ /|\   /|\ \_(}^^A<<<
\ducksay@AddAnimal{snowman}^^A>>>
{  \
    \_[_]_
      (")
   >-( : )-<
    (__:__)}^^A<<<
\ducksay@AddAnimal{hedgehog}^^A>>>
{  \    .\|//||\||.
    \  |/\/||/|//|/|
      /. `|/\\|/||/||
     o__,_|//|/||\||'}^^A<<<
\ducksay@AddAnimal{kangaroo}^^A>>>
{  \
    \ _,'   ___
     <__\__/   \
        \_  /  _\
          \,\ / \\
            //   \\
          ,/'     `\_,}^^A<<<
\ducksay@AddAnimal{rabbit}^^A>>> http://chris.com/ascii/index.php?art=animals/rabbits
{ \     / \`\         __
   \   |  \ `\      /`/ \
    \  \_/`\  \-"-/` /\  \
            |       |  \  |
            (d     b)   \_/
            /       \
        ,".|.'.\_/.'.|.",
       /   /\' _|_ '/\   \
       |  /  '-`"`-'  \  |
       | |             | |
       | \    \   /    / |
        \ \    \ /    / /
         `"`\   :   /'"`
             `""`""`}^^A<<<
\ducksay@AddAnimal{bunny}^^A>>>
{ \
   \      /
      /\ /
       ( )
     .( o ).}^^A<<<
\ducksay@AddAnimal{small-rabbit}^^A>>>
{  \
    \ _//
     (')---.
      _/-_( )o}^^A<<<
\ducksay@AddAnimal{dragon}^^A>>>
{     \                    / \  //\
       \    |\___/|      /   \//  \\
        \   /0  0  \__  /    //  | \ \    
           /     /  \/_/    //   |  \  \  
           @_^_@'/   \/_   //    |   \   \ 
           //_^_/     \/_ //     |    \    \
        ( //) |        \///      |     \     \
      ( / /) _|_ /   )  //       |      \     _\
    ( // /) '/,_ _ _/  ( ; -.    |    _ _\.-~        .-~~~^-.
  (( / / )) ,-{        _      `-.|.-~-.           .~         `.
 (( // / ))  '/\      /                 ~-. _ .-~      .-~^-.  \
 (( /// ))      `.   {            }                   /      \  \
  (( / ))     .----~-.\        \-'                 .~         \  `. \^-.
             ///.----..>        \             _ -~             `.  ^-`  ^-_
               ///-._ _ _ _ _ _ _}^ - - - - ~                     ~-- ,.-~
                                                                  /.-~}^^A<<<
\ducksay@AddAnimal{dog}^^A>>> http://www.ascii-art.de/ascii/def/dogs.txt
{  \     __
    \ .-'\/\
       "\   '------.
     ___/       (  .'_____
    '-----'"""'------"""""'}^^A<<<
\ducksay@AddAnimal{squirrel}^^A>>> http://ascii.co.uk/art/squirrel
{  \           ,;:;;,
    \         ;;;;;
      .=',    ;:;;:,
     /_', "=. ';:;:;
     @=:__,  \,;:;:'
       _(\.=  ;:;;'
      `"_(  _/="`
       `"'``}^^A<<<
\ducksay@AddAnimal{snail}^^A>>>
{  \
    \          .-""-.
      oo      ; .-.  :
       \\__..-: '.__.')._
        "-._.._'.__.-'_.."}^^A<<<
\ducksay@AddAnimal{unicorn}^^A>>> from http://www.ascii-art.de/ascii/uvw/unicorn.txt
{   \
     \       /((((((\\\\
     ---====((((((((((\\\\\
          ((           \\\\\\\
          ( (*    _/      \\\\\\\
            \    /  \      \\\\\\_         __,,__
             |  |   |       </    "------""     ((\\\\
             o_|   /        /                      \ \\\\    \\\\\\\
                  |  ._    (                        \ \\\\\\\\\\\\\\\\
                  | /                       /       /    \\\\\\\     \\
          .______/\/     /                 /       /         \\\
         / __.____/    _/          ___----(       /\
        / / / ________/:______,---'        \     /  \_
       / /  \ \                             \   \ \_  \
      ( <    \ \                             >  /    \ \
       \/      \\_                          / /       > )
                \_|                        / /       / /
                                         _//       _//
                                       /_|       /_|}^^A<<<
\ducksay@AddAnimal{whale}^^A>>> https://asciiart.website//index.php?art=animals/other%20(water)
{ \                |-.
   \     .---._     \ \.--|
    \  /       `-..__)  ,-'
      |     .          /
       \--.__,   .__.,'
        `-.___'._\_.'}^^A<<<
^^A from http://www.ascii-art.de/ascii/s/starwars.txt :
\ducksay@AddAnimal{yoda}^^A>>>
{   \
     \             ____
      \         _.' :  `._
            .-.'`.  ;   .'`.-.
   __      / : ___\ ;  /___ ; \      __
 ,'_ ""--.:__;".-.";: :".-.":__;.--"" _`,
 :' `.t""--.. '<@.`;_  ',@>` ..--""j.' `;
      `:-.._J '-.-'L__ `-- ' L_..-;'
        "-.__ ;  .-"  "-.  : __.-"
            L ' /.------.\ ' J
             "-.   "--"   .-"
            __.l"-:_JL_;-";.__
         .-j/'.;  ;""""  / .'\"-.
       .' /:`. :  :     /.".'';  `.
    .-"  / ;`.".  :    ."."   :    "-.
 .+"-.  : :   ".".". ."."      ;-._   \
 ; \  `.; ; .   "."-"."        : : "+. ;
 :  ;   ; ;  .   ."."    ;     : ;  : \:
 ;  :   ; :     / /     /  ,   ;:   ;  :
: \  ;  :  ;   ; /     :  ,   : ;  /  ::
;  ; :   ; :  ; ;      ;      ;   :   ;:
:  :  ;  :  ;. ;      '      : :  ;  : ;
;\    :   ; : .          ,   ; ;     ; ;
: `."-;   :  ;      .   ;   :  ;    /  ;
 ;    -:   ; :      ,  ,    ;  : .-"   :
 :\     \  :  ;    ,       : \.-"      :
  ;`.    \  ; :   .   ,    ;.'_..--  / ;
  :  "-.  "-:  ;     ,    :/."      .'  :
   \         \ :    :     ;/  __        :
    \       .-`.\        /t-""  ":-+.   :
     `.  .-"    `l    __/ /`. :  ; ; \  ;
       \   .-" .-"-.-"  .' .'j \  /   ;/
        \ / .-"   /.     .'.' ;_:'    ;
         :-""-.`./-.'     /    `.___.'
               \ `t  ._  /
                "-.t-._:'}^^A<<<
\ducksay@AddAnimal{yoda-head}^^A>>>
{   \
     \             ____
      \         _.' :  `._
            .-.'`.  ;   .'`.-.
   __      / : ___\ ;  /___ ; \      __
 ,'_ ""--.:__;".-.";: :".-.":__;.--"" _`,
 :' `.t""--.. '<@.`;_  ',@>` ..--""j.' `;
      `:-.._J '-.-'L__ `-- ' L_..-;'
        "-.__ ;  .-"  "-.  : __.-"
            L ' /.------.\ ' J
             "-.   "--"   .-"
            __.l"-:_JL_;-";.__
         .-j/'.;  ;""""  / .'\"-.
       .' /:`. :  :     /.".'';  `.
    .-"  / ;`.".  :    ."."   :    "-.
 .+"-.  : :   ".".". ."."      ;-._   \}^^A<<<
^^A from https://www.ascii-code.com/ascii-art/movies/star-wars.php
\ducksay@AddAnimal{small-yoda}^^A>>>
{  \
    \
    __.-._
    '-._"7'
     /'.-c
     |  /T
    _)_/LI}^^A<<<
\ducksay@AddAnimal{r2d2}^^A>>>
{  \
    \ ,-----.
    ,'_/_|_\_`.
   /<<::8[O]::>\
  _|-----------|_
 |  | ====-=- |  |
 |  | -=-==== |  |
 \  | ::::|()||  /
  | | ....|()|| |
  | |_________| |
  | |\_______/| |
 /   \ /   \ /   \
 `---' `---' `---'}^^A<<<
\ducksay@AddAnimal{vader}^^A>>>
{  \     _.-'~~~~~~`-._
    \   /      ||      \
       /       ||       \
      |        ||        |
      | _______||_______ |
      |/ ----- \/ ----- \|
     /  (     )  (     )  \
    / \  ----- () -----  / \
   /   \      /||\      /   \
  /     \    /||||\    /     \
 /       \  /||||||\  /       \
/_        \O========O/        _\
  `--...__|`-._  _.-'|__...--'
          |    `'    |}^^A<<<
%</animals>^^A<<<
%
^^A vim: fdm=marker foldmarker=>>>,<<<
