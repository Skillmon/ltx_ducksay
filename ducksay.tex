\documentclass[]{article}

\usepackage{ducksay}
\usepackage{multicol}

\makeatletter
\newcommand*{\availableAnimal}[1]{\@for\cs:=#1\do{%
  \ifx\cs\@empty\else%
    \rlap{\expandafter\ducksay\expandafter[\cs]{\cs}}\hfill\mbox{}\\[1ex]%
  \fi%
}}
\makeatother
\newcommand*{\anml}{\texttt{<animal>}}
\newcommand*{\msg}{\texttt{<message>}}
\newenvironment{codedescription}{%
  \parindent=-3em%
  \parskip=1em%
  \par%
}{}

\begin{document}
\begin{titlepage}%>>>
  \makeatletter
  \centering
  %\mbox{}\vfill
  \Large
    \ducksay[duck,bubble=\huge,msg-align=c]{This is\\ducksay!}\\
  \vfill
  \normalsize
  \hspace*{-2cm}
    \ducksay[cow,bubble=\large]{\ducksay@version}\\
  \small
  \vspace*{-5cm}\hspace*{5cm}
    \ducksay[small-duck,bubble=\normalsize]{But which Version?}
  \mbox{}\hfil
  \vspace{2cm}
  \vfill
  \vfill
  \hspace*{-0cm}
  \large
  \smash{%
    \ducksay[r2d2,bubble=\large]{by Jonathan P. Spratte}}
  \small
    \ducksay[hedgehog,bubble=\normalsize]{Today is \ducksay@date}
  \makeatother
\end{titlepage}%<<<
\tableofcontents
\section{Macros}%>>>
\marginpar{%
  \rlap{%
    \tiny\ducksay[yoda,bubble=\footnotesize,align=t]{Use those, you might}}}
The following macros are available:

\begin{codedescription}
\verb|\ducksay[<options>]{<message>}|\\
  options might include any of the options described in
  section~\ref{sec:options}. Prints an \anml\ saying \msg. \msg\ is not read in
  verbatim. Multi-line \msg s are possible using \verb|\\|.

\verb|\duckthink[<options>]{<message>}|\\
  options might include any of the options described in
  section~\ref{sec:options}. Prints an \anml\ thinking \msg. \msg\ is not read
  in verbatim. It is implemented using regular expressions replacing a \verb|\|
  which is only preceded by \verb|\s*| in the first three lines with \verb|O|
  and \verb|o|. It is therefore slower than \verb|\ducksay|. Multi-line \msg s
  are possible using \verb|\\|.

\verb|\DefaultAnimal{<animal>}|\\
  use the \anml\ if none is given in the optional argument to \verb|\ducksay| or
  \verb|\duckthink|. Package default is \texttt{duck}. You might specify more
  options than \anml\ with this, but it should contain an \anml, otherwise you
  have to specify an \anml\ in each call of \verb|\ducksay|.

\verb|\AddAnimal(*){<animal>}<ascii-art>|\\
  adds \anml\ to the known animals. \texttt{<ascii-art>} is multi-line verbatim
  and therefore should be delimited either by matching braces or by anything
  that works for \verb|\verb|. If the star is given \anml\ is the new default.
  One space is added to the begin of \anml\ (compensating the opening symbol).
  For example, snowman is added with:\\[1ex]
  \begin{minipage}{\linewidth}
\begin{verbatim}
\AddAnimal{snowman}
{  \
    \_[_]_
      (")
   >-( : )-<
    (__:__)}
\end{verbatim}
  \end{minipage}
\end{codedescription}
%<<<
\section{Options}\label{sec:options}%>>>
{\reversemarginpar\marginpar{%
  \vspace*{-2em}\hspace*{-4em}%
  \tiny\ducksay[hedgehog,bubble=\footnotesize,align=t]{Everyone likes options}}}
The following Options are available to \verb|\ducksay| and \verb|\duckthink| and
if not otherwise specified also as package options:

\begin{codedescription}
\anml\\
  One of the animals listed in section~\ref{sec:animals} or any of the ones
  added with \verb|\AddAnimal|. Not useable as package option.

\texttt{bubble=\#1}\\
  use \texttt{\#1} in a group right before the bubble (for font switches). Might
  be used as a package option but not all control sequences work out of the box
  here.

\texttt{body=\#1}\\
  use \texttt{\#1} in a group right before the body (meaning the \anml). Might
  be used as a package option but not all control sequences work out of the box
  here). E.g., to right-align the \anml\ to the bubble, use \verb|body=\hfill|.

\texttt{align=\#1}\\
  use \texttt{\#1} as the vertical alignment specifier given to the
  \texttt{tabular} which is around the contents of \verb|\ducksay| and
  \verb|\duckthink|. Might also be used as a package option.

\texttt{msg-align=\#1}\\
  use \texttt{\#1} for alignment of the rows of multi-line \msg s. It should
  match a \texttt{tabular} column specifier. Default is \texttt{l}. It only
  affects the contents of the speech bubble not the bubble.
\end{codedescription}
  %<<<
\subsection{Package Options}%>>>
The following options are only available as package options during load-time:

\begin{codedescription}
\texttt{animal=\#1}\\
  sets \texttt{\#1} as the default \anml.
\end{codedescription}
%<<<
\section{Defects}%>>>
{\reversemarginpar\marginpar{%
  \tiny\rlap{\ducksay[frog,bubble=\footnotesize,align=t]{Ohh, no!}}}}
\begin{itemize}
  \item no automatic line wrapping
\end{itemize}
%<<<
\section{Dependencies}%>>>
\marginpar{%
  \tiny\rlap{\ducksay[kangaroo,bubble=\footnotesize,align=t]{We rely on you}}}
The package depends on the two packages \texttt{xparse} and \texttt{l3keys2e}
and all of their dependencies.
%<<<
\section{Available Animals}\label{sec:animals}%>>>
\small
\begin{multicols}{2}
\raggedbottom
\availableAnimal{%>>>
  ,duck%
  ,small-duck%
  ,duck-family%
  ,cow%
  ,tux%
  ,pig%
  ,frog%
  ,snowman%
  ,head-in%
  ,hedgehog%
  ,kangaroo%
  ,r2d2%
  ,vader%
  ,yoda-head%
  ,small-yoda%
  ,yoda%
}%<<<
\end{multicols}
%<<<
\clearpage
\thispagestyle{empty}
\bgroup
\Huge
\mbox{}\vfill
\centering
\makebox[0pt]{\duckthink{Who's gonna use it anyway?}}
\vfill
\hfill\smash{\footnotesize\ducksay[small-yoda]{sources at
  \detokenize{https://github.com/Skillmon/ltx_ducksay}}}
\egroup
\end{document}

% vim: fdm=marker foldmarker=>>>,<<<
